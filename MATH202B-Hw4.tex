\documentclass{article}
\usepackage[utf8]{inputenc}
\usepackage{hyperref}
\usepackage{amsthm}
\usepackage{amsmath}
\usepackage{enumitem}
\usepackage{amssymb}
\usepackage{tikz}
\usepackage{tikz-cd}
\usetikzlibrary{positioning}
\usepackage{graphicx}
\usepackage{float}

\title{MATH 202B Hw4}
\author{Jad Damaj}
\date{February 16, 2024}

\usepackage{geometry}
\geometry{a4paper, margin=1in}

\newcommand{\bZ}{\mathbb{Z}}
\newcommand{\bR}{\mathbb{R}}
\newcommand{\bQ}{\mathbb{Q}}
\newcommand{\bF}{\mathbb{F}}
\newcommand{\bC}{\mathbb{C}}
\newcommand{\bN}{\mathbb{N}}
\newcommand{\bH}{\mathbb{H}}
\newcommand{\ex}{\exists}
\newcommand{\fa}{\forall}
\newcommand{\ve}{\varepsilon}
\newcommand{\from}{\leftarrow}
\newcommand{\ifff}{\leftrightarrow}
\newcommand{\cT}{\mathcal{T}}
\newcommand{\vn}{\varnothing}
\newcommand{\vp}{\varphi}
\newcommand{\cB}{\mathcal{B}}
\newcommand{\cC}{\mathcal{C}}
\newcommand{\cL}{\mathcal{L}}
\newcommand{\cM}{\mathcal{M}}
\newcommand{\cP}{\mathcal{P}}
\newcommand{\cF}{\mathcal{F}}
\newcommand{\cA}{\mathcal{A}}
\newcommand{\cG}{\mathcal{G}}
\newcommand{\cD}{\mathcal{D}}
\newcommand{\cO}{\mathcal{O}}
\newcommand{\cN}{\mathcal{N}}
\newcommand{\cQ}{\mathcal{Q}}
\newcommand{\cS}{\mathcal{S}}

\newcommand{\im}{\text{im }}
\newcommand{\Tot}{\text{Tot}}
\newcommand{\dom}{\text{dom}}

\theoremstyle{definition}
\newtheorem{exercise}{Exercise}

\begin{document}

\maketitle

\begin{exercise}
    Let $X$ be a normed vector space over $\bF = \bR$ or $\bF = \bC$. Show that addition and scalar multiplication are continuous from $X \times X$ and $\bF \times X$ to $X$, respectively. Show that $x \mapsto ||x||$ is continuous from $X$ to $[0, \infty)$ and that $|\, ||x||- ||y||\, | \le ||y-x||$ for all $x, y \in X$. 
\end{exercise}

\begin{proof}
    First, to show addition is continuous fix some $(x,y)$ and $\ve>0$. Then letting $\delta = \ve/2$ we see that if $||(x,y)-(x_0, y_0)|| < \delta$ then 
    \[ ||(x+y) - (x_0+y_0)|| = ||(x-x_0)+(y-y_0)|| \le ||x-x_0|| + ||y-y+0|| \le 2\delta = \ve \] 
    Similarly, to see that scalar multiplication is continuous fix some $(\lambda, x)$ and $\ve>0$. Then letting $\delta = \min\{  \ve/(|c| + ||x|| + 1), 1\}$ we compute that 
    \begin{align*}
        ||cx-cx_0|| & = ||cx - cx_0 + cx_0 - c_0x_0|| \\ 
        & \le ||cx-cx_0|| + ||cx_0 - c_0x_0|| \\ 
        & = |c|\, ||x-x_0|| + |c-c_0| \, ||x_0|| \\ 
        & \le |c| \, ||x-x_0|| + |c-c_0| \, (||x-x_0|| + ||x||) \\ 
        & \le |c| \delta + \delta^2 + \delta ||x|| \\ 
        & \le  \delta(|c| + ||x|| + 1) \\ 
        & \le \ve
    \end{align*}
    Next, we show that $|\, ||x|| - ||y||\, | \le ||x-y||$. This follows since $ ||x|| \le ||x-y|| + ||y||$ and so $ ||x||-||y|| \le ||x-y||$. Then, by a symmetric argument we also have $||y||-||x|| \le ||x-y||$ and so the above claim follows. \\ 
    Finally, to show that $x \mapsto ||x||$ is continuous fix some $x$ and $\ve > 0$. Setting $\delta = \ve$, $||x-y|| < \delta$ then by the above we have that $|\, ||x|| - ||y||\, | \le ||x-y|| < \ve$ as well. 

\end{proof}

\begin{exercise}
    Show that $\cL(X,Y)$ is a vector space, and $V \subset X$ is a normed vector space and that the norm $|| \cdot ||_{\cL(X,Y)}$ defined in our text is a norm on this vector space. 
\end{exercise}

\begin{proof}
    First, to show that $\cL(X,Y)$ is a vector space note that there is a natural pointwise addition and scalar multiplication on maps $T: X \to Y$ so it suffices to show that the sum of two bounded operators is bounded and the scalar multiple of a bounded operator is bounded. However, this is immediate since the sum of continuous functions is continuous and multiplication of a continuous function by a constant is also continuous and continuous is equivalent to bounded for linear operators. \\ 
    Next, we show that $||\cdot||_{\cL(X,Y)}$ is a norm. To show it satisfies the triangle inequality, suppose $T_1$ and $T_2$ are operators with $||T_1|| = C_1$ and $||T_2|| =C_2$. Then for each $x$, 
    \[ ||(T_1 + T_2)(x)|| \le ||T_1x|| + ||T_2x|| \le (C_1 + C_2)||x|| \] 
    and so we must have $||T_1 + T_2|| \le C_1 + C_2 = ||T_1|| + ||T_2||$. \\ 
    Similarly, given any $\lambda \neq 0 \in \bF$ we see that for any $C$, $||(\lambda)Tx|| \le \lambda C$ iff $||Tx|| \le C$ and so $||\lambda T|| = \inf\{ \lambda C \colon \fa \, ||Tx|| \le C||x|| \} = \lambda ||T||$. Finally, if $||T||=0$ then we have $||Tx||=0$ for all $x$ and so $T(x)=0$ for all $x$, ie. $T = 0$. 
\end{proof}

\begin{exercise}
    Let $X$ be a normed vector space, and $V \subset X$ a subspace. Show that the closure of $V$ is a subspace of $X$. 
\end{exercise}

\begin{proof}
    Suppose $X$ is a vector space and $V$ a subspace to show that $\overline{V}$ is a subspace note that each element of $\overline{V}$ can be written as the limit of a (not necesarrily distinct) elements of $V$ and so given $x, y \in \overline{V}$ write $x = \lim_{i \to \infty} x_i$ and $y = \lim_{i \to \infty} y_i$ with $x_i, y_i \in V$. Then we have that $x+y = \lim_{i \to \infty}(x_i + y_i) \in \overline{V}$ since each $x_i + y_i \in V$ since it is a subspace. Similarly, using the fact that $\lambda(\lim_{i \to \infty}x_i) = \lim_{i \to \infty}\lambda x_i$ it follows that if $x \in \overline{V}$ and $\lambda \in \bF$ then $\lambda x \in \overline{V}$ as well. 
\end{proof}

\begin{exercise}
    Let $X$ be a normed vector space and $V$ a proper closed subspace. Denote the elements of the quotient space $X/V$ by $x + V$ with $x \in X$, 
\begin{enumerate}[label = (\alph*)]
    \item Show that the quantity $||x+V|| = \inf_{v \in V}||x-v||_X$ is a norm on $X$. 
    \item Show that for any $\ve>0$ there exists $x \in X$ satisfying $||x||_X=1$ such that $||x+V|| > 1 - \ve$. 
    \item Show that the natural projection map $\pi : X /V$ has norm equal to 1. 
    \item Show that if $X$ is complete then so is $X/V$.
\end{enumerate}
\end{exercise}

\begin{exercise}
    Let $X$ be a Banach space. Show that if $X^*$ is separable, then $X$ is separable. 
\end{exercise}

\begin{exercise}
    Let $l^{\infty}$ be the normed vector space of all bounded sequences $x = (x_n \colon n \in \bN)$ with $x_n \in \bF_n$, with the supremum norm. Show that $l^{\infty}$ is not separable. 
\end{exercise}

\begin{exercise}
    Let $V$ be a closed subspace of $X$. By definition a supplement for $V$ is a closed subspace $W$ of $X$ such that $V \cap W = \{0\}$ and $V+W=X$. Show that if $X$ is finite dimensional then $V$ has a supplement. 
\end{exercise}

\end{document}