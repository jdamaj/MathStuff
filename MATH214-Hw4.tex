\documentclass{article}
\usepackage[utf8]{inputenc}
\usepackage{hyperref}
\usepackage{amsthm}
\usepackage{amsmath}
\usepackage{enumitem}
\usepackage{amssymb}
\usepackage{tikz}
\usepackage{tikz-cd}
\usetikzlibrary{positioning}
\usepackage{graphicx}
\usepackage{float}

\usepackage{bbm}

\title{MATH 214 Hw4}
\author{Jad Damaj}
\date{Feb 15, 2024}

\usepackage{geometry}
\geometry{a4paper, margin=1in}

\newcommand{\bZ}{\mathbb{Z}}
\newcommand{\bR}{\mathbb{R}}
\newcommand{\bQ}{\mathbb{Q}}
\newcommand{\bF}{\mathbb{F}}
\newcommand{\bC}{\mathbb{C}}
\newcommand{\bN}{\mathbb{N}}
\newcommand{\bH}{\mathbb{H}}
\newcommand{\ex}{\exists}
\newcommand{\fa}{\forall}
\newcommand{\ve}{\varepsilon}
\newcommand{\from}{\leftarrow}
\newcommand{\ifff}{\leftrightarrow}
\newcommand{\cT}{\mathcal{T}}
\newcommand{\vn}{\varnothing}
\newcommand{\vp}{\varphi}
\newcommand{\cB}{\mathcal{B}}
\newcommand{\cC}{\mathcal{C}}
\newcommand{\cL}{\mathcal{L}}
\newcommand{\cM}{\mathcal{M}}
\newcommand{\cO}{\mathcal{O}}
\newcommand{\cU}{\mathcal{U}}
\newcommand{\bB}{\mathbb{B}}
\newcommand{\bS}{\mathbb{S}}
\newcommand{\bP}{\mathbb{P}}
\newcommand{\cA}{\mathcal{A}}
\newcommand{\cD}{\mathcal{D}}
\newcommand{\cV}{\mathcal{V}}

\newcommand{\im}{\text{im }}
\newcommand{\Tot}{\text{Tot}}
\newcommand{\dom}{\text{dom}}
\newcommand{\Hom}{\text{Hom}}

\newcommand{\Conv}{\mathop{\scalebox{1.5}{\raisebox{-0.2ex}{$\ast$}}}}

\theoremstyle{definition}
\newtheorem{exercise}{Exercise}

\begin{document}

\maketitle

\begin{exercise}[Lee 3.1]
    Suppose $M$ and $N$ are smooth manifolds with or without boundary, and $F: M \to N$ is a smooth map. Show that $dF_p: T_p M \to T_p N$ is the zero map for each $p \in M$ if and only if $F$ is constant on each component of $M$. 
\end{exercise}

\begin{proof}
    First, suppose that $F$ is a smooth map such that $dF_p$ is the zero map for each $p \in M$. We first consider the case where $M$ and $N$ and $\bR^m$ and $\bR^n$. In this case the map $dF_p$ is the matrix $( \partial F^1/ \partial x^j(p))_{ij}$ and so we have the all partials of $F$ are 0 at all points $p$ and so $F$ must be constant on each component. In the case where $M$ and $N$ are arbitrary manifolds, consider charts $(U, \vp)$ and $(V, \psi)$ for the points $p$ and $F(p)$, respectively. If $dF_p$ is 0 for all $p$ then $d(\psi \circ F\circ \vp^{-1})_p = 0$. Then, as this is a map between open subsets of Euclidean space the above proof shows that $\psi \circ F \circ \vp^{-1}$ must be constant on each component and so, since $\psi, \vp$ are homeomorphisms $F$ must be also. Hence, we have shown that for each point $p \in M$, $F$ is constant each component in a neighborhood of $p$ and so it follows that this holds for all of $M$. 
\end{proof}

\begin{exercise}[Lee 3.3]
    Prove that if $M$ and $N$ are smooth manifolds, then $T(M \times N)$ is diffeomorphic to $TM \times TN$. 
\end{exercise}

\begin{proof}
    Suppose $M$ and $N$ are smooth manifolds. Recall that for each $p \in M$ we have an isomorphism of the tangent spaces $T_p(M \times N)$ and $T_p M \times T_p N$ given by $\alpha(v) = (d(\pi_1)_p(v), d(\pi_2)_p(v))$. Define a map $T(M \times N) \to T M \times TN$ by $f (((p,q), v)) = ( (p, d(\pi_1)_{(p,q)} (v)), (q, d(\pi_2)_{(p,q)}(v)))$. Since the map $\alpha$ is an isomorphism it is clear that this map is bijective so to show it is a diffeomorphism we show it is smooth with smooth inverse. \\ 
    First, to check it is smooth at $((p,q), v)$ fix smooth charts $(U, \vp)$ and $(V, \psi)$ containing $p$ and $q$ in $M$ and $N$, respectively. Then, $(\pi^{-1}(U \times V), \widetilde{\vp \times \psi})$ is a smooth chart containing $((p,q), v)$ and $(\pi^{-1}(U) \times \pi^{-1}(V), \tilde{\vp} \times \tilde{\psi})$ is a smooth chart containing its inverse. We compute the transition map to be 
    \begin{align*}
        & (\tilde{\vp} \times \tilde{\psi} \circ F \circ (\widetilde{\vp \times \psi}))^{-1}(\overline{x}, \overline{y}, \overline{v}, \overline{w}) = (\tilde{\vp} \times \tilde{\psi}) \circ F \left( \sum v^i \frac{\partial}{\partial x^i} + \sum w^j \frac{\partial}{\partial y^j} \right)|_{(\vp \times \psi)^{-1}(\overline{x}, \overline{y})} \\ 
         & = (\tilde{\vp} \times \tilde{\psi})\left(d(\pi_1)\left(\sum v^i \frac{\partial}{\partial x^i} + \sum w^j \frac{\partial}{\partial y^j} \right)|_{\vp^{-1}(\overline{x})}, d(\pi_2)\left(\sum v^i \frac{\partial}{\partial x^i} + \sum w^j \frac{\partial}{\partial y^j} \right)|_{\vp^{-1}(\overline{y})}\right) \\
         & = (\tilde{\vp} \times \tilde{\psi})\left(\sum v^i \frac{\partial}{\partial x^i}|_{\vp^{-1}(\overline{x})}, \sum w^j \frac{\partial}{\partial y^j}|_{\vp^{-1}(\overline{y})}\right) \\ 
         & = ((\overline{x}, \overline{v}), (\overline{y}, \overline{w})) 
    \end{align*}
    which is smooth and has smooth inverse (since the inverse of the transition map is the transition map for $F^{-1}$) hence it follows that both $F$ and $F^{-1}$ are smooth, as desired. 
\end{proof}

\begin{exercise}[Lee 3.4]
    Show that $T \bS^1$ is diffeomorphic to $\bS^1 \times \bR$. 
\end{exercise}

\begin{proof}
    For each $p=e^{i \vp}\in \bS^1$ consider the generating element of $T_p \bS^1$ defined by $v_p(f) = (f \circ \gamma)'(0)$ where $\gamma: [-\pi/2, \pi/2] \to \bS^1$ is defined by $\gamma(t) = e^{i(\vp + t)}$. Then each element $w \in T\bS^1$ is equal to $av_p$ for some $a \in \bR$. Now, define the map $F: T\bS^1 \to \bS^1\times \bR$ by $F( (p, av_p)) = (p, a)$. It is clear that $F$ is a bijection.  We show that $F$ is smooth. Fix some $(e^{i \vp}, w) \in T_p M$. We can find some chart $(U, \theta)$ of $\bS^1$ containing $e^{i \vp}$ such that $\theta$ is an angle function. Now, $(\pi^{-1}(U, \tilde{\theta})$ is a chart containing $(e^{i\vp}, w)$ and $(U \times \bR, \theta \times \text{id}_{|bR})$ is a chart containing $F((e^{i\vp}, w))$ and $F(\pi^{-1}(U)) \subseteq U \times \bR$ and so to show $F$ is smooth it is enough to show the transition map is smooth. This follows since 
    \[ (\theta \times \text{id}_{\bR}) \circ F \circ (\tilde{\theta})^{-1}(x, y) = (\theta \times \text{id}_{\bR}) \circ F(e^{ix}, y \cdot v_p) = (\theta \times \text{id}_{\bR})(e^{ix}, y) = (x,y) \] 
    Note that this map has smooth inverse and since its inverse is a transition map for $F^{-1}$ is follows that $F^{-1}$ is smooth as well and so $F$ is a diffeomorphism. 
\end{proof}

\begin{exercise}
    Let $\bS^1 \subseteq \bR^2$ be the unit circle, and let $K \subseteq \bR^2$ be the boundary of the square of side 2 centered at the origin: $K = \{(x,y) \colon \max\{|x|, |y|\} = 1\}$. Show that there is a homeomorphism $F: \bR^2 \to \bR^2$ such that $F(\bS^1) = K$, but there is no diffeomorphism with the same property.  
\end{exercise}

\begin{proof}
    First, to show that such a homeomorphism exists we define a map $F:\bR^2 \to \bR^2$ using polar coordinates, setting $F(0)=0$ and 
    \[ F(r, \theta) = \begin{cases} & \theta \in [\pi/4, 3\pi/4] \\ & \theta \in [3\pi/4, 5\pi/4] \\ & [5 \pi/4, 7 \pi/4] \\ & [7\pi/4] \end{cases} \]
\end{proof}

\begin{exercise}[Lee 3.7]
    Let $M$ be a smooth manifold with or without boundary and $p$ a point of $M$. Let $C_p^{\infty}(M)$ denote the algebra of germs of smooth real-valued functions at $p$, and let $\cD_pM$ denote the vector space of derivations of $C_p^{\infty}(M)$. Define a map $\phi: \cD_pM \to T_pM$ by $(\phi v)f = v([f]_p)$. Show that $\phi$ is an isomorphism.
\end{exercise}

\begin{proof}
    To show that $\phi$ is an isomorphism of vector spaces first note that it is linear since 
    \[(\phi(av+bw))(f) = (av+bw)([f]_p) = av([f]_p) + bw([f]_p) = a \phi(v)(f) + b\phi(w)(f) \] 
    $\phi$ is injective since if $\phi(v)=0$ then $v([f]_p)=0$ for all $f$ and so $v$ must have been the 0 map to begin with. Finally, $\phi$ is surjective since given any $w \in T_pM$ consider $v \in \cD_pM$ defined by $v([f]_p)=w(f)$. Note that this is well defined since if $[f]_p = [g]_p$ then $f$ and $g$ agree on some open set containing $p$ so we must have $w(f) = w(g)$. Then it is clear that $\phi(v)=w$ and so $\phi$ is surjective. Thus it is an isomorphism. 
\end{proof}

\begin{exercise}[Lee 3.8]
    Let $M$ be a smooth manifold with or without boundary and $p \in M$. Let $\cV_p M$ denote the set of equivalence classes of curves starting at $p$ under the relation $\gamma_1 \sim \gamma_2$ of $(f \circ \gamma_1)'(0) = (f \circ \gamma_2)'(0)$ for every smooth real-valued function $f$ defined in a neighborhood of $p$. Show that the map $\psi: \cV_pM \to T_pM$ defined by $(\psi[\gamma])(f) = (f \circ \gamma)'(0)$ is well defined and bijective.  
\end{exercise}

\begin{proof}
    First, to see that $\psi$ is well defined suppose that $\gamma_1 \sim \gamma_2$, ie. $[\gamma_1] = [\gamma_2]$. Then 
    \[ (\psi[\gamma_1])(f) = (f \circ \gamma_1)(0) = (f \circ \gamma_2)(0) = (\psi[\gamma_2])(f) \] 
    for all $f$ so $\psi[\gamma_1] = \psi[\gamma_2]$. By linearity of the deriviative it is clear that $\psi$ is linear as well. \\ 
$\psi$ is injective since if $\psi[\gamma] \equiv 0$ then $(f \circ \gamma)'(0)=0$ for all $f$ and so $\gamma \sim 0$, ie, $[\gamma]= [0]$. \\ 
Finally, to see that $\psi$ is surjective fix some chart $(U, \vp)$ containing $p$. It suffices to show its image contains $\frac{\partial}{\partial x^i}|_p$ since these form a basis for $T_pM$. Now, observe that if $\gamma: [-1,1] \to \bR^n$ is defined by $\gamma_i(t) = (p_1, \ldots, p_i+t, \ldots, p_n)$ then $(f \circ \gamma_i)'(0) = \partial f/\partial x^i (p)$ for each $f: \bR^n \to \bR$. Hence, taking the curve $\vp^{-1} \circ \gamma_i: [-1,1] \to M$, we see that for each $f: M \to \bR$, 
\[ \frac{\partial}{\partial x^i}|_p(f) = \frac{\partial}{\partial x^i}(f \circ \vp^{-1})(p)  = (f \circ \vp^{-1} \circ \gamma_i)'(0) = \psi([\vp^{-1} \circ \gamma_i])(f)  \]
\end{proof}

\end{document}
