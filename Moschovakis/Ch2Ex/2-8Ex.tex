
\begin{exercise}
    Prove that in a complete metric space no open ball in meager. 
\end{exercise}

\begin{proof}
    Suppose towards a contradiction that $B$ is open ball such that $B = \bigcup_n A_n$ where each $A_n$ is nowhere dense. Then we have $B \subset \bigcup \overline{A_n}$. Since $A_1$ is not dense in $B$ we can find some $B_1 \subset B$ such that $\overline{B_1} \subset B \setminus \overline{A_1}$ and such that radius$(B_1) < $. Similarly, given $B_i$ we can find $B_{i+1}$ such that $\overline{B_i} \subset B_i \setminus \overline{A_{i+1}}$ and radius$(B_{i+1})<2^{-(i+1)}$. Now, there exists a unique $b \in \bigcap_{n} \overline{B_n}$ and by construction $b \not\in \overline{A_i}$ for all $i$ and so $b \in B \setminus \bigcup_n A_n$, contradicting our assumption. Hence, $B$ cannot be meager.  
\end{proof}

\noindent
A pointset $P$ has the property of Baire if there is some open poinset $P^*$ such that $P \Delta P^*$ is meager.

\begin{exercise}
    Prove that every Borel poinset has the property of Baire. 
\end{exercise}

\begin{proof}
    We show that the class of poinsets having the property of Baire contains all the open and closed sets, and is closed under negation and countable union. If $P$ is an open pointset, then $P \Delta P = \vn$ is meager so $P$ has the property of Baire. If $P$ is a closed pointset, consider $P^* = P^{\circ}$. $P \Delta P^* = P \setminus P^*$ is meager since it is closed and has empty interior since $(P \setminus P*)^{\circ} \subseteq P^{\circ}$ and $(P \setminus P^*) \cap P^{\circ} = \vn$. Next, suppose $P$ has the property of Baire, we show $\neg P$ does as well. If $P^*$ is an open set such that $P \Delta P^*$ is meager then, since $\neg P \Delta \neg P^* = P \Delta P^*$, $\neg P \Delta \neg P^*$ must be meager as well. Now, it follows that $\neg P \Delta (\neg P^*)^{\circ} \subseteq (\neg P \Delta \neg P^*) \cup (\neg P^*)\setminus (\neg P^*)^{\circ}$ is meager and so $\neg P$ has the property of Baire as well. Finally, we show that if $\{P_n\}_{n \in \omega}$ are such that each $P_i$ has the property of Baire $\bigcup_i P_i$ doees as well. Suppose we have open $P_i^*$ such that $P_i \Delta P_i^*$ is meager. Then, $\bigcup_i P_i \Delta \bigcup_i P_i^* \subseteq \bigcup_i (P_i \Delta P_i^*)$ is meager and so $\bigcup_{i} P_i$ has the property of Baire as well.  
\end{proof}

\begin{exercise}
    Prove that for every poinset $P \subseteq \cX$, there is an $F_{\sigma}$ set $\tilde{P} \supseteq P$ such that if $A \subseteq P^*\setminus P$ is Borel, then $A$ is meager. 
\end{exercise}

\begin{proof}
    Suppose $P$ is an arbitary pointset. Consider the set 
    \[ D = \{x \colon \text{ for all neighboroods of} N \text{ of }x \, N\cap P \text{ is not meager}\} \] 
    $D$ is open since if $x \in \neg D$, then there is some $B_r(x)$ such that $B_r(x) \cap P$ is meager. Now, for all $y \in B_{r/2}(x)$, $B_{r/2}(y) \subseteq B_r(x)$ and so $B_{r/2}(y) \cap P$ is meager. Hence, $B_{r/2}(x) \subseteq \neg D$. \\
    Next, observe that $P\setminus D$ must be meager since each point $x \in P \setminus D$ must be contained in some basic open neighborhood, $N_x$ such that $N_x \cap P$ is meager and so, since there are only countable many such neighborhoods, the union $\bigcup_{x \in P \setminus D} (N_x \cap P)$ is a countable union of meager sets and so must be meager. Further, this set must be contained in some $F_{\sigma}$ meager set $W$ since if $P \setminus D = \bigcup A_n$ where each $A_n$ is no where dense, then $P \setminus D \subset \bigcup \overline{A_n}$ as well. \\
    We define $P^* = D \subseteq W$ and claim that for any Borel set $A \subseteq P^* \subseteq P$, $A$ must be meager. To see why this is true suppose $A$ is such a Borel set. Then, there is some open set $A^*$ such that $A \Delta A^*$ is meager. $A^* \cap P \subseteq (A^* \setminus A)$ is meager and so $A^* \cap D = \vn$. However, this implies $A^* \subseteq W \cup (A^* \setminus A)$ which is meager and so we must have $A^* = \vn$. Hence, $A$ must be meager. 
\end{proof}

\begin{exercise}
    Prove that the collection of pointsets with the property of Baire is closed under the operation $\cA$; in particular $\undertilde{\Sigma}_1^1$ sets have the property of Baire. 
\end{exercise}

\begin{proof}
    Let $\cC$ be the $\sigma$-algebra of open sets and let $J$ be the $\sigma$-ideal consisting of the meager sets. The previous exercise showed that $J$ is regular from above relative to $\cC$ and so, by the Approximation theorem, the collection of all pointsets that are in $\cC$ modulo $J$, ie. the sets with the property of Baire, is closed under $\cA$. In particular, since this class contains all Borel sets, it must contain all $\undertilde{\Sigma}_1^1$ sets as well. 
\end{proof}